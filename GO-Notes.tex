%Schriftgr��e, Layout, Papierformat, Art des Dokumentes
\documentclass[10pt,oneside,a4paper]{scrreprt}

%Einstellungen der Seitenr�nder
\usepackage[left=1cm,right=1cm,top=2cm,bottom=2cm]{geometry}



\parindent 0em
\parskip 0.3em

\usepackage{verbatim} 
\usepackage{listings}
\usepackage{amsmath}
\usepackage{amssymb}
\newcommand\boxit[1]{%
  \hspace*{7mm}\frame{\parbox{500pt}{ \vspace*{3mm}\hspace*{2mm} \parbox{480pt}{#1}\vspace*{3mm}}}
}

\newcommand\boxtwo[1]{%
  \hspace*{7mm}\frame{\parbox{238pt}{ \vspace*{3mm}\hspace*{2mm} \parbox{230pt}{#1}\vspace*{3mm}}}
}


\begin{document}
    % \section*{Relationen}
    % \subsubsection*{Funktionen}
    % \boxit{
    % injektiv: \hfill $ \forall x,y \in Y : f(x) = f(y) \Rightarrow x = y$ \\
    % Linkseindeutig: Jedes Element der Zielmenge ist h\"ochstens einmal Funktionswert
    % }
    
    \section*{Graphentheorie}
    
    Graph: $G = (V,E)$\\
    Baum: $|E| = |V| - 1$\\
    Bipartit: $G = (V_1 \cup V_2, E)$,
    G ist bipartit $\Leftrightarrow$ G enth\"allt keinen Kreis gerader l\"ange\\
    
    \subsection*{Planarit\"at}
    f\"ur ebene Darstellungen gelten: $n-m+f=2$ ,n=Knoten, m=Kanten, f=Fl\"achen\\
    if $n \geq 3$ $3n-6$ Kanten h\"ochsten\\
    if $n \geq 3$ hat h\"ochstens $g \geq 3$, g = Umfang des Graphen???? $max\{g(n-2)/(g-2)m n-1\} Kanten$ \\
    ein Graph ist planar $\Leftrightarrow$ kein subgraph von G ist hom\"oomorph zu $k_5$, $k_{3,3}$
    
    \subsection*{Datenstrukturen}
    Adjazenzmatrix, $n \cdot n$, immer symetrisch\\
    $a_{ij} = 1$ falls $v_iv_j \in E$, $0$ sonst\\
    Inzidenzmatrix, $n\cdot m$, $e_{ij} = 1 $ wenn $v_i$ mit $e_j$ inzidiert, $0$ sonst\\
    spaltensume immer $2$, reihensumme = grad des Knoten\\
    
    \subsection*{Netzwerke}
    Floyd-Warshal (S.288)\\
    \boxit{
    Abstand f\"ur alle Knoten $O(|V|^3)$\\
    siehe auch Dijkstra (S.289)\\

    for k=1 to n do: $ d(u,w) = min(d^{k-1}(u,w), d^{k-1}(u,v_k) + d^{k-1}(v_k,w))$\\
    Mit jeder Iteration gucken ob es einen k\"urzeren Weg \"uber den Knoten $v_k$ gibt
    }
    
    
    Kurskal (S.291)\\
    \boxit{
    min. Spannb\"aume
    }
    
    Ford-Fulkerson (S.293)\\
    \boxit{
      
    }
    
\end{document}